\section{Conjuntos y Funciones}

Una de las deficiones más importantes es la defición de conjunto. Esta es una de las definiciones más intuitivas, pero a la vez más difíciles de definir.
La definición de conjunto que se utilizará es la siguiente: \textit{Un conjunto es una colección no ordenada de elementos distintos.}\footnotemark[1]


\begin{Def}{Conjuntos}{Def_Conjuntos}
    \begin{itemize}{}
        \item  El conjunto sin elementos ($\{ \}$) se denota por $\emptyset$.
        \item Si $x \in B $ para todo $x \in A $, entonces $A$ es un subconjunto de $B$ y se denota por $A \subseteq B$.
        \item La unión de dos conjuntos $A$ y $B$ es el conjunto $A \cup B = \{ x | \hspace{0.1cm} x \in A \lor x \in B \}$.
        \item La intersección de dos conjuntos $A$ y $B$ es el conjunto $A \cap B = \{ x | \hspace{0.1cm} x \in A \land x \in B \}$.
    \end{itemize}
\end{Def}

\vspace{1cm}

Ahora se definirá el concepto de función. Una función es una relación entre dos conjuntos, Dado un conjunto $A$ y un conjunto $B$, una función $f$ de $A$ en $B$ es
una regla que asigna a cada elemento $x \in A$ un único elemento $f(x) \in B$. Se denota por $f: A \rightarrow B$.
\begin{enumerate}
    \item   Además, $A$ es el dominio de $f$ y $B$ es el codominio de $f$. El rango de $f$ es el conjunto de todos los valores que toma $f$ y se denota por $R_{f}$. (Es decir, $R_{f} = \{ f(x) | \hspace{0.1cm} x \in A \}$).\vspace{0.1cm}
    \item   Una función $f: A \rightarrow B$ es inyectiva si $f(x) = f(y) \Rightarrow x = y$ para todo $x,y \in A$.\footnotemark[2]
    \item   Una función $f: A \rightarrow B$ es sobreyectiva si para todo $x \in B$, existe algún $a \in A$, tal que, $f(a)=B$.\footnotemark[3]
    \item   Una función es biyectiva si es inyectiva y sobreyectiva.
\end{enumerate}

\footnotetext[1]{\textit{Todo es un conjunto}}
\footnotetext[2]{\textit{Igualmente se puede definir de la siguiente manera: $f$ es inyectiva si $x \neq y \Rightarrow f(x) \neq f(y)$ para todo $x,y \in A$.}}
\footnotetext[3]{$\forall x \in B \hspace{0.1cm} \exists a \in A \hspace{0.1cm} |\hspace{0.1cm} f(a) = b$}

\newpage

\section{Propiedades del valor absoluto}
\vspace{0.3cm}
\begin{multicols}{2}
\begin{enumerate}[parsep =6pt, itemsep = 15pt]
    \item $|x| \geq 0$ para todo $x \in \mathbb{R}$.
    \item $|x| = |-x|$
    \item $-|x| \leq x \leq |x|$
    \item $|a\cdot b| =  |a|\cdot|b|$
    \item $\mathlarger{\frac{|a|}{|b|} = \left|\frac{a}{b}\right|}, \hspace{0.1cm} b \neq 0$
    \item $|a| \leq b \Leftrightarrow -b \leq a \leq b$
    \item $|a+b| \leq |a| + |b|$
    \item $||a|-|b|| \leq |a-b|$
\end{enumerate}
\end{multicols}

\begin{Problemas}{}{}
Demuestre las siguientes propiedades del valor absoluto:
\begin{itemize}[parsep =6pt, itemsep = 15pt]
    \item $|a| \leq b \Leftrightarrow -b \leq a \leq b$
    \item $|a+b| \leq |a| + |b|$
    \item $||a|-|b|| \leq |a-b|$
\end{itemize}
\end{Problemas}

