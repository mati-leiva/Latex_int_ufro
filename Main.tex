\documentclass[a4paper]{article}
\usepackage{amssymb}
\usepackage{amsthm}
\usepackage{amsmath}
\usepackage{latexsym}
\usepackage{fancyhdr}
\usepackage{relsize}
\usepackage[T1]{fontenc}
\usepackage{lmodern}
\usepackage[utf8]{inputenc}
\usepackage[spanish]{babel}
\usepackage[most]{tcolorbox}
\usepackage{hyperref}
\hypersetup{
    colorlinks=true,
    linkcolor=blue,
    filecolor=magenta,      
    urlcolor=cyan,
    pdftitle={Overleaf Example},
    pdfpagemode=FullScreen,
    }

\urlstyle{same}



\usepackage[left = 1cm, right = 1cm,top = 1.25cm, bottom=1.50cm,% Set the height and width of the paper
includehead,
nomarginpar,% We don't want any margin paragraphs
textwidth=10cm,% Set \textwidth to 10cm
headheight=20pt,% Set \headheight to 25pt to accommodate the multiline header
]{geometry}



% % % % % % % % % % % % % % % % % % % % % % % % % % % % % % % % % % % % % % % % % % % % % 

%* Comandos y cosas así

\newtcolorbox{CajaTitulo}[1]{
colback=yellow!5!white,%
colframe=white!50!black,%
title={#1},%
}

\newtcolorbox{CajaSolucion}{
    colback=yellow!5!white,%
    colframe=white!50!black,%
}

\newcommand{\cambio}[2]{
        \begin{cases}
        #1 \\ 
        #2  
        \end{cases}
}

\newcommand{\casillaCambio}[3]{
\begin{minipage}{0.4\linewidth}
    \cambio{#1}{#2}
\end{minipage}
\begin{minipage}{0.5\linewidth}
    $\implies#3$
\end{minipage}
}

\newcommand{\diff}{\mathop{}\!d}

\newcommand{\PorPartes}[4]
{
    \left[
      \begin{alignedat}{2}
      u       &=  #1             \quad & \diff v&= #3\\
      \diff u&=  #2             \quad & v &= #4
      \end{alignedat}\,
    \right]
}

\newcommand{\Brackets}{\Big|}


\newcommand{\I}{\mathcal{I}}

\newcommand{\myHearts}
 {$\color{purple}{\heartsuit}\kern-2.5pt$}

\newcommand{\Porlo}{
    \therefore}

\newcommand{\Flecha}[1]{\overset{#1}{\longrightarrow}}



% % % % % % % % % % % % % % % % % % % % % % % % % % % % % % % % % % % % % % % % % % % % %  

\pagestyle{plain}

\begin{document}




\begin{titlepage}
    \begin{center}
        \vspace*{1cm}

        \LARGE\textbf{Resolución de Integrales}

        \vspace{0.5cm}
        

        \vspace{1cm}

        \textsc{IntegralesUfro}

        \vspace{5cm}

        \includegraphics[width=0.4\textwidth]{Imagenes/Logo.png}

        \vspace{0.8cm}
    \end{center}
\end{titlepage}

\newpage

\pagestyle{fancy}
\fancyhead{} % clear all header fields
\fancyhead[RO,LE]{\textbf{Resolución de Integrales.}}


\tableofcontents

\include{Problemas/primera_pag}

\section{Soluciones}    
Esta es una breve introducción a lo que serán las soluciones de los problemas planteados en la sección anterior, 
en esta sección se mostrarán las soluciones de los problemas planteados en el semestre, y se mostrarán los pasos a seguir para resolverlos, 
además de mostrar los trucos que se pueden utilizar para resolverlos. \\

Es importante mencionar que las soluciones que se mostrarán en esta sección, no son las únicas soluciones posibles, si tienen alguna duda o sugerencia, no duden en 
contactarme a mi cuenta de instagram: \href{https://www.instagram.com/integralesufro/}{@integralesufro}. \\

Finalmente, para que esta introduccion no sea tan aburrida, les dejo una historia conmovedora: \\  \\


\begin{minipage}{0.4\linewidth}
    \includegraphics[width=0.7\textwidth]{Imagenes/gato.jpg}
\end{minipage}
\begin{minipage}{0.5\linewidth}
    Había una vez un gatito matemático llamado Fibonacci, cuyo amor por las secuencias perfectas era proporcional a su curiosidad felina. 
    Saltaba entre números primos y raíces cuadradas, pero su fascinación máxima residía en la mágica Razón Áurea. Deslizándose por la espiral,
    Fibonacci encontró un universo geométrico donde los triángulos se multiplicaban siguiendo su famosa serie. Calculó los ángulos divinos y,
    en ese instante, su fama felina creció exponencialmente entre los matemáticos, ¡miau-tiplicando su legado para siempre! \myHearts
\end{minipage}


\begin{CajaTitulo}{\begin{center}\subsection{Solución P1}\end{center}}
\vspace{1cm}
    Sea $\I$ la integral a resolver, entonces, tenemos que: \[ \I =  \int_{0}^{\mathlarger{\frac{\pi}{2}}}\frac{\sqrt[3]{\tan{(x)}}dx}{{(\sin{(x)}+\cos{(x)})}^2} \cdot \frac{\sec{(x)}^2}{\sec{(x)}^2}\]
    Entonces obtenemos la siguiente integral: \[\I =  \int_{0}^{\mathlarger{\frac{\pi}{2}}}\frac{\sqrt[3]{\tan{(x)}}\cdot \sec{(x)}^2 dx}{{(\tan{(x)}+1)}^2}\]
    Haciendo el siguiente cambio de variable: \\ \\

    \[\cambio{\tan{(x)} = u}{\sec{(x)}^2 dx = du}  \Porlo \I = \int_{0}^{\infty}\frac{\sqrt[3]{u}}{{(u+1)}^2} du  \hspace{0.1cm} \rightarrow \cambio{\sqrt[3]{u} = a \vspace{0.1cm}}{\mathlarger{\frac{\sqrt[3]{u}}{3u}}du = da} \Porlo\I = \int_{0}^{\infty}\frac{3u^3}{{(1+u^3)}^2}du\]


    Luego realizando integración por partes, obtenemos: \[ \int_{0}^{\infty}\frac{3x^3}{{(1+x^3)}^2}dx = \PorPartes{x}{\diff x}{\frac{3u^2}{{(u^3+1)}^2}\diff u}{\frac{-1}{(u^3 +1)}} \implies \frac{-x}{{(x^3+1)}^2} \Brackets_0^\infty  + \int_{0}^{\infty} \frac{1}{x^3 + 1}dx \] \\

    Ahora para realizar la Integral $\mathlarger{\int_{0}^{\infty} \frac{1}{x^3 + 1}dx}$, hacemos el siguiente truco, llamemos a la integral $\I$, entonces: \\ \\

    \[\cambio{\mathlarger{\frac{1}{x}} = u \vspace{0.2cm}}{\mathlarger{\frac{1}{x^2}dx}= -du} \Porlo \I = \int_{0}^{\infty}\frac{u}{u^3 +1}du \rightarrow 2\I = \int_{0}^{\infty}\frac{u}{u^3 +1}du + \int_{0}^{\infty} \frac{1}{x^3 + 1}dx\]

    Entonces la integral $\I = \mathlarger{\frac{1}{2}\int_{0}^{\infty}\frac{1}{x^2 -x +1}dx}$, ahora para resolver, completamos el cuadrado en el denominador, obteniendo lo siguiente: \[ \I = \frac{1}{2}\int_{0}^{\infty}\frac{1}{{(x-\frac{1}{2})}^2 + \frac{3}{4}}dx \] \\

    Después los siguientes pasos es transformar la integral, para obtener un arco tangente, lo que resulta en la siguiente expresión: \\

    % Necesito que hagas una arco tangente y la completes

    \[\lim_{s \to \infty}  \frac{1}{\sqrt{3}} \arctan(x) \Brackets_\mathlarger{\frac{-1}{\sqrt{3}}} ^\mathlarger{s} = \frac{2\pi}{3\sqrt{3}}   \]

    \vspace{2cm}
\end{CajaTitulo}

\begin{CajaTitulo}{\begin{center}\subsection{Solución P2}\end{center}}
    \vspace{1cm}
    Para resolver esta integral usaremos bastantes trucos, asi que vamos a ir paso a paso. Primero, vamos a hacer la siguiente integración por partes:

    \[\int_{0}^{\infty}\frac{dx}{ {(x^4 +1)}^n} = \I(n), \hspace{0.5cm}\PorPartes{\frac{1}{{(x^4 +1)}^n}}{\frac{-n\cdot 4x^3}{{(x^4 +1)}^{n+1}}}{dx}{x} = \frac{x}{{(x^4 +1)}^n} 
    \Brackets_{0}^{\infty} + 4n\int_{0}^{\infty}\frac{x^4}{{(x^4 +1)}^{n+1}}dx\]
    \\


    \[\I(n) = 4n\int_{0}^{\infty}\frac{x^4}{{(x^4 +1)}^{n+1}}dx = 4n\int_{0}^{\infty}\frac{x^4 + 1 - 1}{{(x^4 +1)}^{n+1}}dx = 4n\cdot \I(n) - 4n \cdot\I(n+1)\] 
    \\
Aquí usamos la siguiente formula de recurrencia: 
    \[\I(n) = 4n \cdot \I(n) - 4n \cdot \I(n+1) \implies \I(n+1) =  \frac{4n-1}{4n}\cdot\I(n)\]

    \[\therefore \I(3) = \frac{4\cdot 2 - 1}{4\cdot2} \cdot \frac{4\cdot1 - 1}{4\cdot1}\cdot\I(1)\]

    Ahora nos queda calcular $\I(1)$, para ello usamos la siguiente sustitución:\footnote{Un compañero en instagram resolvió una integral similar, aquí les dejo el link de la solución: \href{https://www.instagram.com/p/Cr7I89yg4Ww/?img_index=2}{IntegralesQueHablan solución.} }

    \[\int_{0}^{\infty}\frac{dx}{1 + x^4} = \int_{0}^{\infty}\frac{dx}{\mathlarger{\frac{1}{x^2}} + x^2}\cdot\frac{1}{x^2} \hspace*{0.2cm}\cambio{\mathlarger{\frac{1}{x}} = u \vspace{0.1cm}}{\mathlarger{\frac{1}{x^2}}dx = -du} \rightarrow  \hspace{0.2cm}\Porlo\I = \int_{0}^{\infty}\frac{du}{u^2 + \mathlarger{\frac{1}{u^2}}} = \I = \int_{0}^{\infty}\frac{u^2}{u^4 + 1 }du \]
    \\


    \[ 2\I = \int_{0}^{\infty}\frac{x^2 + 1}{x^4 +1}dx = \int_{0}^{\infty}\frac{1 + \mathlarger{\frac{1}{x^2}}}{ {(x - \mathlarger{\frac{1}{x}})}^2 + 2 }\cdot\frac{1}{x^2}\hspace{0.1cm} dx \cambio{x - \mathlarger{\frac{1}{x}} = u \vspace{0.13cm}}{1 + \mathlarger{\frac{1}{x^2}}dx = du}\]

    \[2\I = \int_{-\infty}^{\infty}\frac{1}{u^2 + 2}du \hspace{1cm}\Porlo\I = \int_{0}^{\infty}\frac{1}{u^2 + 2}du = \frac{\pi}{2\sqrt{2}}\]
    \\

    Finalmente tendríamos que, $\I(3) = \mathlarger{\frac{4\cdot 2 - 1}{4\cdot2} \cdot \frac{4\cdot1 - 1}{4\cdot1}\cdot  \frac{\pi}{2\sqrt{2}} = \frac{21\pi}{64\sqrt{2}}}$

    \vspace{6cm}
 
\end{CajaTitulo}

\begin{CajaTitulo}{\begin{center}\subsection{Solución P3}\end{center}}
    \vspace{1cm}
    \[\int_{0}^{\mathlarger{\frac{\pi}{2}}}\frac{\ln(a \cdot \sec(x))}{1 + \sin^2(x)}dx \cdot \frac{\sec^2(x)}{\sec^2(x)} = \int_{0}^{\mathlarger{\frac{\pi}{2}}}\frac{\ln(a \cdot \sec(x))\sec^2(x)}{\sec^2(x) + \tan^2(x)}dx =  \int_{0}^{\mathlarger{\frac{\pi}{2}}}\frac{(\ln(a) + \ln(\sec(x)))}{1 + 2\tan^2(x)}\cdot\sec^2(x)dx \]

    \[\cambio{\tan(x) = u}{dx \cdot \sec^2(x) = du}\rightarrow\int_{0}^{\infty}\frac{\ln(a) + \frac{1}{2}\ln(u^2 + 1)}{1 + 2u^2}du = \int_{0}^{\infty}\frac{\ln(a)}{1 + 2u^2}du + \frac{1}{2}\int_{0}^{\infty}\frac{\ln(u^2 + 1)}{2u^2 + 1}du \]

    \[\cambio{u\sqrt{2} = x}{du = \mathlarger{\frac{dx}{\sqrt{2}}}} \rightarrow \frac{1}{\sqrt{2}} \left(       \int_{0}^{\infty}\frac{\ln(a)}{1+x^2}dx  \hspace{0.1cm} +  \frac{1}{2}\int_{0}^{\infty}\frac{\ln(\mathlarger{\frac{x^2}{2}} + 1)}{x^2 +1}dx\right) = \frac{\pi}{2\sqrt{2}}\ln(a) + \frac{1}{2\sqrt{2}}\int_{0}^{\infty}\frac{\ln(x^2 +2) - \ln(2)}{x^2 +1}dx\]

    \[\frac{\pi}{2\sqrt{2}}\ln(a) - \frac{\pi\ln(2)}{4\sqrt{2}} + \frac{1}{2\sqrt{2}}\int_{0}^{\infty}\frac{\ln(x^2 +2)}{x^2 +1}dx\footnote{Resultado que nos servirá más adelante, recordar!!} \hspace{0.2cm} \cambio{x = \tan{\theta}}{dx = \sec^2(\theta )d\theta} \rightarrow  \frac{1}{2\sqrt{2}}\int_{0}^{\mathlarger{\frac{\pi}{2}}}\ln(\tan^2(\theta) + 2)d\theta\]
    \\
    Ahora usaremos un nuevo truco, que es la derivación bajo el signo de la integral\footnote{Aquí puedes encontrar un ejemplo más sencillo: \url{https://www.youtube.com/watch?v=wsfvX4Mtrho&}}, para ello definimos la función $\I(a)$ como:

    \[\I(a)\footnote{Notar que $\I(0) = 0$. \url{https://www.youtube.com/watch?v=iNaiq_IETEs}} = \int_{0}^{\mathlarger{\frac{\pi}{2}}}\ln(\tan^2(\theta) + a)d\theta \rightarrow \I'(a) = \int_{0}^{\mathlarger{\frac{\pi}{2}}}\frac{\partial}{\partial a }\ln(\tan^2(\theta) + a)d\theta = \int_{0}^{\mathlarger{\frac{\pi}{2}}}\frac{1}{\tan^2(\theta) + a}d\theta  \]

    \[\I'(a) = \int_{0}^{\mathlarger{\frac{\pi}{2}}}\frac{1}{\tan^2(\theta) + a}\cdot\frac{\sec^2(\theta)}{1 + \tan^2(\theta)}d\theta \hspace{0.2cm}\cambio{\tan \theta = u} {\sec^2(\theta)d\theta = du} \rightarrow \int_{0}^{\infty} \frac{1}{(u^2 + a)(u^2 + 1)}du\]

    Realizando fracciones parciales, tenemos que:

    \[\I'(a) = \frac{1}{a-1}\int_{0}^{\infty}\frac{1}{u^2 +1}du - \frac{1}{a-1}\int_{0}^{\infty}\frac{1}{u^2 + a}du\]

    \[\I'(a) = \frac{1}{a-1}\cdot \frac{\pi}{2} - \frac{1}{(a-1)\sqrt{a}}\cdot\frac{\pi}{2} \rightarrow \frac{\pi}{2}\frac{1}{a-1}\left(1 - \frac{1}{\sqrt{a}}\right) = \frac{\pi}{2}\frac{1}{\sqrt{a}(\sqrt{a}+1)}\]

    Luego, integrando respecto a $a$ tenemos que:

    \[\I(a) = \frac{\pi}{2}\int\frac{1}{\sqrt{a}(\sqrt{a}+1)}da = \cambio{\sqrt{a} = v}{\mathlarger{\frac{1}{2\sqrt{a}}}da = dv} \rightarrow \pi\int\frac{1}{v+1}dv = \pi\ln(\sqrt{a}+1) + C\]

    Finalmente, como $\I(0) = 0$, tenemos que $C = 0$, por lo que:

    \[\I(a) = \pi\ln(\sqrt{a}+1) \implies \I(2) = \pi \ln(\sqrt{2} +1)\]

    Dandonos como resultado final:

    \[\frac{\pi}{2\sqrt{2}}\ln(a) - \frac{\pi\ln(2)}{4\sqrt{2}} + \frac{1}{2\sqrt{2}}\I(2) = \frac{\pi}{2\sqrt{2}}\ln(a) - \frac{\pi}{2\sqrt{2}} \ln(\sqrt{2}) + \frac{\pi}{2\sqrt{2}}\ln(\sqrt{2} +1)\]

    \vspace{3cm}
\end{CajaTitulo}

\begin{CajaTitulo}{\begin{center}\subsection{Solución P4}\end{center}}
    \vspace{1cm}
    \[\int_{0}^{\infty}\frac{\ln({(ax)}^2 + 1)}{x^2 + b^2}dx = \cambio{ax = u}{dx = \mathlarger{\frac{du}{a}}}\rightarrow \frac{1}{a}\int_{0}^{\infty}\frac{\ln{(u^2 +1)}}{\mathlarger{\frac{u^2}{a^2} + b^2}}\cdot\frac{a^2}{a^2}du \rightarrow a\int_{0}^{\infty}\frac{\ln{(u^2 +1)}}{u^2+ {(ab)}^2}du\]

    \[a\int_{0}^{\infty}\frac{\ln{(u^2 +1)}}{u^2+ {(ab)}^2}du = \cambio{u = x \cdot (ab)}{du = dx \cdot (ab)} \rightarrow a^2 b \int_{0}^{\infty} \frac{\ln(1 + x^2 \cdot {(ab)}^2)}{{(ab)}^2(1 + x^2)}dx = \frac{1}{b} \int_{0}^{\infty} \frac{\ln{\left({(ab)}^2(\mathlarger{\frac{1}{{(ab)}^2}} + x^2) \right)}}{1 + x^2}dx \]

    \[\frac{2}{b}\int_{0}^{\infty}\frac{\ln(ab)}{x^2 + 1}dx + \frac{2}{b}\int_{0}^{\infty}\frac{\ln{\left(x^2 + \mathlarger{ \frac{1}{{(ab)}^2}}  \right)}}{x^2 + 1}dx\footnote{Aquí utilizamos el resultado de la integral anterior, cuando usamos el truco de feymann.} = \frac{\pi}{b}\ln(ab) + \frac{\pi}{b}\ln{\left(\frac{1}{ab} +1\right)} = \frac{\pi}{b}\ln{\left({ab} +1\right)}\]
    \\
    
    Bueno, este resultado fue bastante corto, para rellenar un poco más, dejemos algunas integrales, que pueden ser útiles para el futuro:

    \[\int_{0}^{\mathlarger{\frac{\pi}{2}}}\frac{\ln(\sec(x))}{1 + a\sin^2(x)}dx\footnote{En realidad nunca más usaremos esta integral, solo la quería mostrar jejejeje.} = \frac{\pi}{2\sqrt{a+1}}\left(\ln(\sqrt{a+1}+1) - \ln{(\sqrt{a+1}\hspace{0.1cm})} \right)\]
    \vspace{4cm}
\end{CajaTitulo}

\begin{CajaTitulo}{\begin{center}\subsection{Solución P5}\end{center}}

\vspace{0.5cm}

\[\int_{0}^{\infty}\frac{dx}{(x^4 + a_1)(x^4 + a_2)\cdots(x^4 + a_n)} \rightarrow \frac{1}{(x^4 + a_1)(x^4 + a_2)\cdots(x^4 + a_n)} = \frac{A}{x^4 + a_1} + \frac{B}{x^4 + a_2} + \cdots + \frac{Z}{x^4 + a_n}\]
\\
Multiplicando por el denominador de la izquierda, obtenemos que: 
\[1 = A(x^4 + a_2)(x^4 + a_3)\cdots(x^4 + a_n) + B(x^4 + a_1)(x^4 + a_3)\cdots(x^4 + a_n) + \cdots + Z(x^4 + a_1)\cdots(x^4 + a_{n-1}) \rightarrow \cambio{x = \sqrt[4]{a_1}\cdot\sqrt{i} }{}\]

\[1 = A(a_2 - a_1)(a_3 - a_1)\cdots(a_n - a_1) + 0 + \cdots + 0 \implies A = \frac{1}{(a_2 - a_1)(a_3 - a_1)\cdots(a_n - a_1)}\]

Repitiendo todos los pasos anteriores, pero ahora con $x = \sqrt[4]{a_2}\cdot\sqrt{i}$, obtenemos que: 

\[1 = 0 + B(a_1 - a_2)(a_3 - a_2)\cdots(a_n - a_2) + \cdots + 0 \implies B = \frac{1}{(a_1 - a_2)(a_3 - a_2)\cdots(a_n - a_2)}\]

Y así sucesivamente, hasta llegar a que:

\[Z = \frac{1}{(a_1 - a_n)(a_2 - a_n)\cdots(a_{n-1} - a_n)}\]

Ya que $a_i \neq a_j$ para $i \neq j$, entonces podemos escribir que:

\[\int_{0}^{\infty}\frac{dx}{(x^4 + a_1)(x^4 + a_2)\cdots(x^4 + a_n)} = A\cdot\int_{0}^{\infty}\frac{dx}{x^4 + a_1} + \cdots + Z\cdot\int_{0}^{\infty}\frac{dx}{x^4 + a_n}\]

\[A\cdot\int_{0}^{\infty}\frac{dx}{x^4 + a_1} = \cambio{x = u\sqrt[4]{a_1}}{dx = du \cdot \sqrt[4]{a_1}} \rightarrow \frac{A}{\sqrt[4]{{(a_1)}^3}}\int_{0}^{\infty}\frac{du}{u^4 + 1} \footnote{Esta integral la hicimos en el problema 2!! Pan comido!.} = \frac{A}{\sqrt[4]{{(a_1)}^3}} \cdot \frac{\pi}{2\sqrt{2}} \]

Repetimos este proceso para cada una de las integrales, y obtenemos que la siguiente expresión es la solución al problema:

\[\frac{\pi}{2\sqrt{2}}\left(\frac{1}{\sqrt[4]{{(a_1)}^3}(a_2 - a_1)(a_3 - a_1)\cdots(a_n - a_1)} + \cdots + \frac{1}{\sqrt[4]{{(a_n)}^3}(a_1 - a_n)(a_2 - a_n)\cdots(a_{n-1} - a_n)}\right)\]
\\

Un ejemplo de esto es $\mathlarger{\int_{0}^{\infty}\frac{dx}{(x^4 + 4)(x^4 + \pi)}} $, entonces la solución es: 

\[ \frac{\pi}{2\sqrt{2}}\left(\frac{1}{\sqrt[4](4^3)(\pi - 4)} + \frac{1}{\sqrt[4](\pi^3)(4 - \pi) }  \right) \approx 0,0908\]              
\\
Lo que concuerda con la solución de Wolfram Alpha:
\[\int_0^{\infty } \frac{1}{\left(x^4+4\right) \left(x^4+\pi \right)} \, dx   = \frac{-2 \sqrt{2} \pi ^{1/4}+\pi }{8 (-4+\pi )}\approx 0.0908648\]

\end{CajaTitulo}


\begin{CajaTitulo}{\begin{center}\subsection{Solución P6}\end{center}}
    \vspace{0.5cm}
\[\int_{0}^{2\pi}\frac{d\theta}{1 - 2a\cos(\theta) + a^2}\footnote{Aquí asumí que $a$ era mayor que 1, después habría que hacer varios cambios, en cuanto valores absolutos y cosas así, tarea para el lector muejejej.} = \cambio{\theta - \pi = x}{d\theta = dx} \rightarrow \int_{-\pi}^{\pi}\frac{dx}{1 - 2a\cos(x + \pi) + a^2} = \int_{-\pi}^{\pi}\frac{dx}{1 + 2a\cos(x) + a^2} = \cambio{\tan(\frac{x}{2}) = u}{dx = \mathlarger{\frac{2du}{1 + u^2}}} \]
\\
\[\int_{-\infty}^{\infty}\frac{2du}{2a(1-u^2) + (a^2 + 1)(1+u^2)} = \int_{-\infty}^{\infty}\frac{2du}{(a^2 +1)u^2 - 2au^2 + (a^2 +2a +1 )} = \int_{-\infty}^{\infty}\frac{2du}{{(a - 1)}^2u^2  + {(a+1)}^2}\]
\\
\[\frac{2}{{(a - 1)}^2}\int_{-\infty}^{\infty}\frac{du}{u^2  + \mathlarger{\frac{{(a+1)}^2}{{(a - 1)}^2}}} = \frac{2}{{(a - 1)}^2}\cdot\frac{{(a - 1)}}{{(a+1)}}\cdot\arctan\left(\frac{x(a-1)}{a+1}\right)\Brackets_{-\infty}^{\infty} = \frac{2\pi}{a^2 -1}\]
\vspace{4cm}
\end{CajaTitulo}


\begin{center}
    \section{Integrales Segundo Semestre 2023}
\end{center}


\begin{CajaTitulo}{\begin{center}\subsection{Solución P7}\end{center}}
    \vspace{0.7cm}

    \[\int_{0}^{\infty}e^{-x^2}\cos(x)dx \implies \int_{0}^{\infty}e^{-x^2}\cos(ax)dx \:=\: \I(a) , \hspace{0.3cm}I(0) = \frac{\sqrt{\pi}}{2} \]
    \[\frac{d}{da}\int_{0}^{\infty}e^{-x^2}\cos(ax)dx \:= \: \I'(a) \implies \int_{0}^{\infty}\frac{\partial}{\partial a}e^{-x^2}\cos(ax)dx  = \int_{0}^{\infty}e^{-x^2}\sin(ax)(-x)dx\]
    \\
    Integrando por partes, obtenemos que:
    \[\PorPartes{\sin(ax)}{\cos(ax) \cdot (a) dx}{e^{-x^2}(-x)dx}{\frac{e^{-x^2}}{2}} =   \frac{e^{-x^2}}{2} \sin(ax) \Brackets_{0}^{\infty}  - \frac{a}{2}\int_{0}^{\infty}e^{-x^2}\cos(ax)dx     \]
    \[\I'(a) = - \frac{a}{2}\int_{0}^{\infty}e^{-x^2}\cos(ax)dx = - \frac{a}{2}\I(a), \hspace{1cm} \Porlo \hspace{0.2cm} \frac{\I'(a)}{\I(a)} = \frac{-a}{2}  \]
    \[\int \frac{\I'(a)}{\I(a)}da = \int \frac{-a}{2} da \implies \ln(\I(a)) = \frac{-a^2}{4} + C, \: \: \I(a) = e^{\frac{-a^2}{4} + C}\]

    Para $\I(0)$, tenemos que: $\I(0) = e^{\frac{-0^2}{4} + C} = e^{C} = \frac{\sqrt{\pi}}{2} \implies C = \ln(\mathlarger{\frac{\sqrt{\pi}}{2}})$ Entonces, $\I(a) = e^{\frac{-a^2}{4} + \ln(\frac{\sqrt{\pi}}{2})} = \mathlarger{ \frac{\sqrt{\pi}}{2}e^{\mathlarger{\frac{-a^2}{4}}}}$

    \[\I(a) = \int_{0}^{\infty}e^{-x^2}\cos(ax)dx = \frac{\sqrt{\pi}}{2}e^{\mathlarger{\frac{-a^2}{4}}}\]
    \[\I(1) = \int_{0}^{\infty}e^{-x^2}\cos(x)dx =  \frac{\sqrt{\pi}}{2}e^{\mathlarger{\frac{-1}{4}}}  \hspace{1cm}\square\]

    Demostración de la integral de Gaussiana:
    \\

    Consideremos la siguiente función:

    \[\I(a) = \int_{0}^{\infty}\frac{ e^{-a^2(1+x^2)}}{1 + x^2}dx \hspace{0.2cm} \Flecha{\mathlarger{\frac{d}{da}}} \: \: \I'(a) = \int_{0}^{\infty}e^{-a^2(1+x^2)}(-2a) = (-2a)e^{-a^2}\int_{0}^{\infty}e^{-(ax)^2} dx \]
    \[\I'(a) = (-2a)e^{-a^2} \int_{0}^{\infty}e^{-{(ax)}^2} dx \hspace{0.2cm}   \cambio{ax = u}{dx \cdot a = du} = (-2) e^{-a^2} \int_{0}^{\infty}e^{-{u}^2}du\]
    \[\int_{0}^{\infty}\I'(a)da = (-2) \int_{0}^{\infty}e^{-a^2}da \int_{0}^{\infty}e^{-{u}^2}du \implies \I(\infty) - \I(0) = -2 \left(\int_{0}^{\infty}e^{-{u}^2}du\right)^2\]
    \[\frac{-\pi}{2} = -2\left(\int_{0}^{\infty}e^{-{u}^2}du\right)^2 \hspace{0.2cm}\]
    \[ \Porlo  \int_{0}^{\infty}e^{-x^2}dx = \frac{\sqrt{\pi}}{2}\]
\end{CajaTitulo}

\begin{CajaTitulo}{\begin{center}\subsection{Solución P7}\end{center}}
    \vspace{0.7cm}
    \[\int_{0}^{\infty}xe^{-x^3}dx \:\cambio{x^3 = u}{ dx = du \mathlarger{\frac{\sqrt[3]{u^2}}{3}}} = \int_{0}^{\infty}e^u u^{-\frac{1}{3}}du = \frac{1}{3}\varGamma\left(\frac{2}{3}\right)\]
    Una integral que en si, tiene una solución bastante sencilla, pero que podemos sacarle ``\textit{jugo}'' resolviéndola de otra forma.\\ 
    \vspace{0.3cm}

    \[\int_{0}^{\infty}\frac{e^{-(1+x^3)a^3}}{1+x^3}dx = \I(a) \: \Flecha{\mathlarger{\frac{d}{da}}} \: \: -3a^2\int_{0}^{\infty}e^{-(1+x^3)a^3}dx = \I'(a) \]
    \[-3a^2e^{-a^3}\int_{0}^{\infty}e^{{(ax)}^3}dx = \I'(a) \: \cambio{ax = u}{dx \cdot a = du} = -3a\cdot e^{-a^3}\int_{0}^{\infty}e^{-x^3}dx = \I'(a)\]
    \[-3\int_{0}^{\infty}e^{-a^3}a\:da\int_{0}^{\infty}e^{-u^3}du = \I(\infty) - \I(0) \rightarrow -3\int_{0}^{\infty}e^{-a^3}a\:da\int_{0}^{\infty}e^{-u^3}du = -\frac{2\pi}{3\sqrt{3}} \]
    \[\int_{0}^{\infty}e^{-a^3}a\:da\int_{0}^{\infty}e^{-u^3}du = \frac{2\pi}{3\sqrt{3}}\]
    \[\int_{0}^{\infty}e^{-u^3}du =  \implies \int_{0}^{\infty}e^{-u^3}du  = \frac{2\pi}{3\sqrt{3} \:\varGamma\left(\frac{2}{3}\right)}\]
    \[\Porlo \varGamma\left(\frac{1}{3}\right)\varGamma\left(\frac{2}{3}\right) =  \frac{2\pi}{9\sqrt{3}}  =\varGamma\left(1 - \frac{2}{3}\right)\varGamma\left(\frac{2}{3}\right) \]
    Con esto, podemos notar una relación entre el producto de dos funciones $\varGamma$, esta relación es llama la \textbf{fórmula de reflexión de Euler}.  
        \[\varGamma(z)\varGamma(1-z) = \frac{\pi}{\sin(\pi z)}\]
    \vspace{0.3cm}
    \[\int_{0}^{\infty}\frac{e^{-(1+x^n)a^n}}{1+x^n}dx = \I(a) \: \Flecha{\mathlarger{\frac{d}{da}}} -n(a^{n-1})e^{-a^n}\int_{0}^{\infty}e^{-{(ax)}^n}dx = \I'(a) \]
    \[\int_{0}^{\infty}(a^{n-1})e^{-a^n}da\int_{0}^{\infty}e^{-{(ax)}^n}dx = \frac{1}{n}\int_{0}^{\infty}\frac{1}{1+x^n}dx\]
    \[\int_{0}^{\infty}(a^{n-1})e^{-a^n}da\int_{0}^{\infty}e^{-u}u^{\mathlarger{\frac{1}{n}}-1}n \:du = \frac{1}{n}\int_{0}^{\infty}\frac{1}{1+x^n}dx\]
    \[\varGamma(n)\cdot \varGamma\left(\frac{1}{n}-1\right)\frac{1}{n} = \frac{1}{n}\int_{0}^{\infty}\frac{1}{1+x^n}dx\]
    \[\varGamma(n)\cdot \varGamma\left(1-n\right) = \frac{1}{n}\int_{0}^{\infty}\frac{1}{1+x^n}dx\cambio{x = \tan^{\mathlarger{\frac{2}{n}}}(u) \vspace{0.2cm}}{dx = {\frac{2}{n^2}}  du\cdot{\tan(u)}^{\mathlarger{\frac{2}{n}}-1}\sec^2(u)} = \frac{2}{n^2}\int_{0}^{\mathlarger{\frac{\pi}{2}}}{\tan(u)}^{\mathlarger{\frac{2}{n}}-1}\cdot \:\frac{\sec^2(u)}{\sec^2(u)}\cdot du\]
    \[\frac{2}{n^2}\int_{0}^{\mathlarger{\frac{\pi}{2}}}{\tan(u)}^{\mathlarger{\frac{2}{n}}-1}du \footnote{De aquí no sé como seguir, se los dejo a ustedes, tengo fé.}\]

    
\end{CajaTitulo}

\end{document}