\begin{CajaTitulo}{\begin{center}\subsection{Solución P3}\end{center}}
    \vspace{1cm}
    \[\int_{0}^{\mathlarger{\frac{\pi}{2}}}\frac{\ln(a \cdot \sec(x))}{1 + \sin^2(x)}dx \cdot \frac{\sec^2(x)}{\sec^2(x)} = \int_{0}^{\mathlarger{\frac{\pi}{2}}}\frac{\ln(a \cdot \sec(x))\sec^2(x)}{\sec^2(x) + \tan^2(x)}dx =  \int_{0}^{\mathlarger{\frac{\pi}{2}}}\frac{(\ln(a) + \ln(\sec(x)))}{1 + 2\tan^2(x)}\cdot\sec^2(x)dx \]

    \[\cambio{\tan(x) = u}{dx \cdot \sec^2(x) = du}\rightarrow\int_{0}^{\infty}\frac{\ln(a) + \frac{1}{2}\ln(u^2 + 1)}{1 + 2u^2}du = \int_{0}^{\infty}\frac{\ln(a)}{1 + 2u^2}du + \frac{1}{2}\int_{0}^{\infty}\frac{\ln(u^2 + 1)}{2u^2 + 1}du \]

    \[\cambio{u\sqrt{2} = x}{du = \mathlarger{\frac{dx}{\sqrt{2}}}} \rightarrow \frac{1}{\sqrt{2}} \left(       \int_{0}^{\infty}\frac{\ln(a)}{1+x^2}dx  \hspace{0.1cm} +  \frac{1}{2}\int_{0}^{\infty}\frac{\ln(\mathlarger{\frac{x^2}{2}} + 1)}{x^2 +1}dx\right) = \frac{\pi}{2\sqrt{2}}\ln(a) + \frac{1}{2\sqrt{2}}\int_{0}^{\infty}\frac{\ln(x^2 +2) - \ln(2)}{x^2 +1}dx\]

    \[\frac{\pi}{2\sqrt{2}}\ln(a) - \frac{\pi\ln(2)}{4\sqrt{2}} + \frac{1}{2\sqrt{2}}\int_{0}^{\infty}\frac{\ln(x^2 +2)}{x^2 +1}dx\footnote{Resultado que nos servirá más adelante, recordar!!} \hspace{0.2cm} \cambio{x = \tan{\theta}}{dx = \sec^2(\theta )d\theta} \rightarrow  \frac{1}{2\sqrt{2}}\int_{0}^{\mathlarger{\frac{\pi}{2}}}\ln(\tan^2(\theta) + 2)d\theta\]
    \\
    Ahora usaremos un nuevo truco, que es la derivación bajo el signo de la integral\footnote{Aquí puedes encontrar un ejemplo más sencillo: \url{https://www.youtube.com/watch?v=wsfvX4Mtrho&}}, para ello definimos la función $\I(a)$ como:

    \[\I(a)\footnote{Notar que $\I(0) = 0$. \url{https://www.youtube.com/watch?v=iNaiq_IETEs}} = \int_{0}^{\mathlarger{\frac{\pi}{2}}}\ln(\tan^2(\theta) + a)d\theta \rightarrow \I'(a) = \int_{0}^{\mathlarger{\frac{\pi}{2}}}\frac{\partial}{\partial a }\ln(\tan^2(\theta) + a)d\theta = \int_{0}^{\mathlarger{\frac{\pi}{2}}}\frac{1}{\tan^2(\theta) + a}d\theta  \]

    \[\I'(a) = \int_{0}^{\mathlarger{\frac{\pi}{2}}}\frac{1}{\tan^2(\theta) + a}\cdot\frac{\sec^2(\theta)}{1 + \tan^2(\theta)}d\theta \hspace{0.2cm}\cambio{\tan \theta = u} {\sec^2(\theta)d\theta = du} \rightarrow \int_{0}^{\infty} \frac{1}{(u^2 + a)(u^2 + 1)}du\]

    Realizando fracciones parciales, tenemos que:

    \[\I'(a) = \frac{1}{a-1}\int_{0}^{\infty}\frac{1}{u^2 +1}du - \frac{1}{a-1}\int_{0}^{\infty}\frac{1}{u^2 + a}du\]

    \[\I'(a) = \frac{1}{a-1}\cdot \frac{\pi}{2} - \frac{1}{(a-1)\sqrt{a}}\cdot\frac{\pi}{2} \rightarrow \frac{\pi}{2}\frac{1}{a-1}\left(1 - \frac{1}{\sqrt{a}}\right) = \frac{\pi}{2}\frac{1}{\sqrt{a}(\sqrt{a}+1)}\]

    Luego, integrando respecto a $a$ tenemos que:

    \[\I(a) = \frac{\pi}{2}\int\frac{1}{\sqrt{a}(\sqrt{a}+1)}da = \cambio{\sqrt{a} = v}{\mathlarger{\frac{1}{2\sqrt{a}}}da = dv} \rightarrow \pi\int\frac{1}{v+1}dv = \pi\ln(\sqrt{a}+1) + C\]

    Finalmente, como $\I(0) = 0$, tenemos que $C = 0$, por lo que:

    \[\I(a) = \pi\ln(\sqrt{a}+1) \implies \I(2) = \pi \ln(\sqrt{2} +1)\]

    Dandonos como resultado final:

    \[\frac{\pi}{2\sqrt{2}}\ln(a) - \frac{\pi\ln(2)}{4\sqrt{2}} + \frac{1}{2\sqrt{2}}\I(2) = \frac{\pi}{2\sqrt{2}}\ln(a) - \frac{\pi}{2\sqrt{2}} \ln(\sqrt{2}) + \frac{\pi}{2\sqrt{2}}\ln(\sqrt{2} +1)\]

    \vspace{3cm}
\end{CajaTitulo}